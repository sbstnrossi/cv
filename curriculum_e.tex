\documentclass[a4paper,10pt, sans]{article}
\usepackage[utf8x]{inputenc}
\usepackage[spanish]{babel}
\usepackage{hyperref}
\usepackage{verbatim}
\usepackage{graphicx}
\usepackage{float}
\usepackage{wrapfig}
\usepackage{LTXtable}
%\usepackage{amsmath,amssymb,amsfonts,latexsym,cancel}
\usepackage{multicol}      %para usar varias columnas
  %uso:
  %\begin{multicols}{2}
    %texto
  %\end{multicols}
\usepackage{marvosym}

\setlength{\columnsep}{4mm}    %separación de columnas
%\usepackage{pslatex}
%\hoffset -0.75in        %cambio el margen horizontal izquierdo
%\voffset -0.9in          %cambio el margen vertical superior
\textwidth 17cm        %ancho de texto. defaut: 14cm
\textheight 25cm        %alto de texto. default: 19cm
\topmargin -1cm        %agranda el margen superior. default: 3cm
\oddsidemargin -1cm        %agranda el margen izquierdo. default: 2.5cm (4.5 si no aparece esta instrucción)
\pagestyle{empty} %{myheadins}  %numeración de pág: sin / arriba
\parskip 2mm          %X mm entre párrafos
%\parindent 0mm          %elimina sangría

\begin{document}
  

  \begin{wrapfigure}{r}{4cm}
    \includegraphics[height=4cm]{seba_4x4.jpg}
  \end{wrapfigure}

  \sffamily
  
  \begin{Huge}
    Sebastián Rossi
  \end{Huge}
  \\ \\
  \hspace*{0.5cm} \textit{CURRICULUM VITAE} \\
  \hspace*{0.5cm} {\textit{Updated: 07/06/2022}}
  
  \begin{tabular}{rl}
    \vspace{0.5cm} \\
    \large\Mobilefone & \textbf{+549 3402 507471} \\
    \large\Letter & \textbf{srossi@inti.gob.ar}
  \end{tabular}

  \vspace{0.5cm}
  \large
  \begin{table}[H]
  \begin{tabularx}{\textwidth}{r X}
    %%%%%%%%%%%%%%%%%%%%%%%%%%%%%%%%%%%%%%%%%%%%%%%%%%%%%%%%%%%%%%%%%%%%%%%%%%%%
    \textbf{Personal Details} & {} \\ [1ex]
      {} & Birth date: 15-11-1988 \\ [1ex]
      {} & Nationality: Argentinian - Italian \\ [1ex]
      {} & Address: Maipú 782, (2107) Álvarez, Santa Fe, Argentina \\ \\ \hline \\

    %%%%%%%%%%%%%%%%%%%%%%%%%%%%%%%%%%%%%%%%%%%%%%%%%%%%%%%%%%%%%%%%%%%%%%%%%%%%
    
    \textbf{Education} & {} \\ [1ex]
       (2020 - ) & \textbf{Doctorado en Ingeniería (Doctor of Engineering)}, FCEIA, Universidad Nacional de Rosario (UNR). Thesis: Sensing and control of pneumatic planting systems. Supervisors: Ignacio Rubio Scola, Gastón Bourges.\\ [1ex]
       (2008 - 2014) & \textbf{Ingeniero Electrónico (Electronic Engineer)}. FCEIA, UNR.\\ \\ \hline \\
    %%%%%%%%%%%%%%%%%%%%%%%%%%%%%%%%%%%%%%%%%%%%%%%%%%%%%%%%%%%%%%%%%%%%%%%%%%%%
    \textbf{Employment} & {}\\ [1ex]
      (2013 - ) & Instituto Nacional de Tecnología Industrial (INTI - National Institute of Industrial Technology). Department of Industrial Products Engineering - Rosario.\\ \\
            
        {} & \hspace{2cm} \textit{Development of new capabilities.} \\ [1ex]
        
        (2013) & Temperature distribution and thermal stabilization time measurements in gauge blocks calibration laboratory to determine their effects on the reported uncertainty. \\ [1ex]
        (2013 - 2014) & Software development for data acquisition by serial port of a gauge block comparator. Programmed in Python.\\ [1ex]
        (2014 - 2015) & Development and implementation of a multichannel recording system commanded by Wi-Fi with Raspberry Pi boards. Application: Seed impact detection. \\ [1ex]
        (2015) & Scilab script development for determining impact times of different kinds of seeds from piezoelectric sensor signal and seed metering systems evaluation according to ISO 7256. \\ [1ex]
        (2016) & Cortex M4 based platform programming for synchronized multi channel piezoelectric sensors for seed impact detection. Programmed in C.\\ [1ex]
        (2016 - 2017) & Documentation drafting to integrate the strain gauge measurement service into the quality management system. \\ [1ex]
        (2017 - 2019) & Air-drill system test bench experimentation for different levels of air speed, seed density, distributor head configuration and exit pipe lengths with soybean seeds. \\ [1ex]
        (2018) & Cortex M0 based modular platform development with wireless communication for data acquisition of multi channel piezoelectric sensors and pressure transmitters. Programmed in C. Two layers PCBs designed with Altium.\\ [1ex]
        (2019 - 2020) & Image processing software development in python using OpenCV library for seed trajectory detection in high frame rate videos for evaluating seed meters and seed delivery tubes in a test bench over a shaker. R script for results analysis. \\ [1ex]
        (2020 - 2021) & Static bench experiment for uncertainty evaluation of seed and trajectory detection systems with impact plate and image processing. \\ [1ex]        
        
  \end{tabularx}
  \end{table}
  
  \begin{table}[H]
  \centering
  \begin{tabularx}{\textwidth}{r X}  
  		(2022 - ) & Static bench experiment for uncertainty evaluation of seed detection systems with different materials of impact plate and infrared sensor array.\\ \\
        {} & \hspace{2cm} \textit{Work done for clients.} \\ [1ex]
        (2015 - 2016) & Evaluation of an in-line seed drill using piezoelectric sensors both in test bench and in field (soybean) based on ISO 7256 Part 2. Client: Indecar. \\  [1ex]

        (2015 - ) & Evaluation of precision seed meters and seed delivery tubes with image processing and piezoelectric sensors based on ISO 7256 Part 1, both in static and dynamic (shaker) test benches. Clients: Siembra Neumática, Pla, Agrometal. \\ [1ex]
        (2016 - ) & Bonding of strain gauges and measurement of deformations in steel structures. Clients: Sola y Brusa, Ejército Argentino, Ombú, Pla, Superwalter, Acerías 4C, Randon, Hermann, Crucianelli, Agrometal, Palfinger. \\ \\ \hline \\

    \textbf{Conference papers} & {}\\ [1ex]
      {} & \textit{Experimental evaluation of a precision seed meter.} (in Spanish) V CAIM 2016. Santiago del Estero, Argentina. UNSE- FCEyT- ISBN: 978-987-1676-63-7. pp: 881-891. \\  [1ex]
      {} & \textit{Tests on a precision seed meter.} (in Spanish) X Jornadas de Ciencia y Tecnología. Sede de Gobierno. UNR. Rosario, Santa Fe. \\  [1ex]
      {} & \textit{Performance evaluation of a precision seed meter by means of filming and image processing.} (in Spanish) VII CAIM 2020. San Nicolás, Argentina. \\  [1ex]
      {} & \textit{Incidence evaluation of different factors in the distribution of soybean seeds in a pneumatic transport system.} (in Spanish)  VII CAIM 2020. San Nicolás, Argentina. \\ \\ \hline \\
      
    \textbf{Courses} & {}\\ [1ex]
    {} & Measurement uncertainty evaluation. 16 hs. INTI Rosario. \\  [1ex]
    {} & Solidworks. 24 hs. Disegno Soft S.R.L. \\  [1ex]
    {} & PIC microcontrollers programming in C language. 30 hs. Postgraduate school of FCEIA, UNR. \\  [1ex]
    {} & Digital systems synthesis in FPGA. 30 hs. Postgraduate school of FCEIA, UNR. \\  [1ex]
    {} & RTOS on 32 bits based microcontrollers. 30 hs. Postgraduate school of FCEIA, UNR. \\  [1ex]
    {} & Experiment design. 30 hs. Postgraduate school of FCAgr, UNR. \\  [1ex]
    {} & Multivariate analysis. 50 hs. Postgraduate school of FCAgr, UNR. \\  [1ex]
    {} & Introduction to design and analysis of industrial experiments. Graduate school of FCECON, UNR. \\  [1ex]
    {} & Digital Image Processing. 70 hs. Postgraduate school of FCEIA, UNR. \\  [1ex]
    {} & Modelling and simulation of dynamic systems. 70 hs. Postgraduate school of FCEIA, UNR. \\ \\ \hline \\
      
    \textbf{Languages} & {} \\ [1ex]
    {} & Spanish, native language. \\ [1ex]
    {} & English, intermediate level. \\ \\ \hline \\
      
    \textbf{Driver's license} & {} \\ [1ex]
    {} & Argentinian, A3 and B2 classes.\\
    
\vspace{5cm}
    
    
    
    
        
  \end{tabularx}
  \end{table}
  
\end{document}
